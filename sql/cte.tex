\chapter{CTEs: Common Table Expressions}

\section{¿Qué es un CTE?}

Un Common Table Expression (CTE) es una forma de crear una consulta
temporal nombrada que puedes usar en la misma consulta SQL.
Los CTEs hacen que las consultas sean más legibles y modulares,
especialmente en escenarios complejos.

\section{Ventajas de los CTEs}

\begin{itemize}
    \item \textbf{Legibilidad:} Los CTEs hacen que las consultas sean más fáciles de leer y mantener.
    \item \textbf{Reutilización:} Puedes reutilizar un CTE en la misma consulta.
    \item \textbf{Modularidad:} Puedes dividir una consulta compleja en partes más pequeñas y manejables.
    \item \textbf{Alternativa a las subconsultas:} Ayudan a evitar consultas anidadas difíciles de leer.
\end{itemize}

\section{Sintaxis de un CTE}

\begin{lstlisting}[language=SQL]
WITH cte_name AS (
    SELECT column1, column2
    FROM table_name
    WHERE condition
)
SELECT column1
FROM cte_name
WHERE another_condition;
\end{lstlisting}

\begin{examplebox}{Ejemplo de CTE}
    Crea una consulta que muestre los empleados cuyo salario es mayor que el salario promedio de su departamento.
\end{examplebox}

\vspace{10pt}

\begin{lstlisting}[language=SQL]
WITH DepartmentAverage AS (
    SELECT department_id, AVG(salary) AS avg_salary
    FROM employees
    GROUP BY department_id
)
SELECT e.emplyee_id, e.name, e.salary, e.department_id
FROM employees e
JOIN DepartmentAverage d
ON e.department_id = d.department_id
WHERE e.salary > d.avg_salary;
\end{lstlisting}

En este ejemplo, la consulta \texttt{DepartmentAverage} calcula el salario
promedio por departamento. Luego, la consulta principal
se une con el CTE \texttt{DepartmentAverage} para filtrar los empleados.

De esta forma, los CTEs hacen que la consulta sea más legible y modular.

\section{Cuándo usar CTEs}

\begin{itemize}
    \item \textbf{Dividir consultas complejas:} Cuando una consulta tiene múltiples niveles de anidación o subconsultas difíciles de leer
    \item \textbf{Evitar redundancia:} Si necesitas reutilizar una subconsulta en la misma consulta.
    \item \textbf{Mejorar la legibilidad:} Para hacer que la consulta sea más fácil de entender y mantener.
    \item \textbf{Recursión:} Cuando trabajas con datos jerárquicos o necesitas realizar cálculos iterativos (recursividad).
\end{itemize}

\section{CTEs Recursivos}

Un CTE recursivo es un CTE que se referencia a sí mismo en la definición.
Es útil para trabajar con datos jerárquicos, como estructuras de árbol
(ej. una tabla de empleados y sus jefes)

\subsection{Sintaxis de un CTE Recursivo}

\begin{lstlisting}[language=SQL]
WITH RECURSIVE cte_name AS (
    -- Anchor Member: Define la base de la recursión
    SELECT column1, column2
    FROM table_name
    WHERE condition

    UNION ALL

    -- Recursive Member: Define cómo la recursión avanza
    SELECT column1, column2
    FROM table_name
    JOIN cte_name
    ON table_name.column = cte_name.column
)
SELECT *
FROM cte_name;
\end{lstlisting}

\begin{examplebox}{Ejemplo de CTE Recursivo}
    Crea una consulta que muestre la jerarquía de empleados en una tabla de empleados.
\end{examplebox}

\vspace{10pt}

\begin{lstlisting}[language=SQL]
WITH RECURSIVE EmployeeHierarchy AS (
    -- Anchor Member: el CEO no tiene jefe
    SELECT employee_id, name, manager_id, 1 AS level
    FROM employees e
    WHERE manager_id IS NULL

    UNION ALL

    -- Recursive Member: Empleados con jefe
    SELECT e.employee_id, e.name, e.manager_id, eh.level + 1
    FROM employees e
    JOIN EmployeeHierarchy eh
    ON e.manager_id = eh.employee_id
)
SELECT *
FROM EmployeeHierarchy
ORDER BY level;
\end{lstlisting}

\section{Buenas Prácticas con CTEs}

\begin{itemize}
    \item \textbf{Usa nombres descriptivos:} Elige nombres significativos para los CTEs.
    \item \textbf{Divide consultas complejas:} Usa un CTE por cada <<bloque lógico>> de la consulta.
    \item \textbf{Evita CTEs innecesarios:} Si una consulta se puede escribir sin CTEs, no los uses por estética.
    \item \textbf{Cuidado con el rendimiento:} Un CTE no se almacena físicamente y se evalúa cada vez que se referencia. Para reutilización intensiva, considera una tabla temporal.
    \item \textbf{Limita la recursividad:} Asegúrate de establecer una condición de terminación en los CTEs recursivos para evitar bucles infinitos.
\end{itemize}

