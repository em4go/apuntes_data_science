\chapter{Window Functions}\label{ch:window_functions}

Las window functions son una herramienta para realizar cálculos que
dependen de un conjunto de filas relacionadas, conocidas como
una ventana.

A diferencia de las funciones de agregación como \texttt{SUM()} o
\texttt{AVG()}, las window functions no agrupan los resultados en
una sola fila. En su lugar, devuelven un valor para cada fila.

\section{Qué son las window functions}

Una \textbf{window function} es una función que opera sobre un
conjunto de filas definidas por una cláusula \texttt{OVER}. A
este conjunto de filas se le conoce como <<ventana>>.

\subsection{Comparación con funciones de agregación}

\begin{examplebox}{Ejemplo con función de agregación}
    Media de salarios por departamento con \texttt{GROUP BY}
\end{examplebox}

Devuelve una fila por departamento y se pierde el detalle de cada empleado.
\vspace{5pt}

\begin{lstlisting}[language=SQL]
SELECT department_id, AVG(salary) as avg_salary
FROM employees
GROUP BY department_id;
\end{lstlisting}


\begin{examplebox}{Ejemplo con window function}
    Rango de salarios por departamento con \texttt{OVER}
\end{examplebox}

Devuelve todas las filas, pero incluye la media de salarios por
departamento junto a cada empleado.
\vspace{5pt}

\begin{lstlisting}[language=SQL]
    SELECT employee_id, department_id, salary,
    AVG(salary) OVER (PARTITION BY department_id) as avg_salary
    FROM employees;
\end{lstlisting}


\section{¿Cuándo se usan las window functions?}

\begin{itemize}
    \item \textbf{Ranking y análisis relativo:}
    \begin{itemize}
        \item Determinar el puesto de un empleado dentro de su departamento.
        \item Comparar ventas de un mes con el promedio anual.
    \end{itemize}
    \item \textbf{Cálculos acumulativos:}
    \begin{itemize}
        \item Sumar ingresos mes a mes de un cliente.
        \item Obtener totales acumulativos en un periodo de tiempo.
    \end{itemize}
    \item \textbf{Agregados no destructivos:}
    \begin{itemize}
        \item Mostrar el promedio o el máximo sin agrupar los resultados.
    \end{itemize}
    \item \textbf{Comparaciones entre filas:}
    \begin{itemize}
        \item Calcular diferencias entre el salario de un empleado y
        el promedio de su departamento.
    \end{itemize}
    \item \textbf{Análisis de series temporales:}
    \begin{itemize}
        \item Calcular la variación de ventas de un producto en el tiempo.
    \end{itemize}
\end{itemize}

\section{Sintaxis de las window functions}

Una función agregada tiene la siguiente estructura:

\begin{lstlisting}[language=SQL]
<function>(<column>) OVER (
    [PARTITION BY <column>]
    [ORDER BY <column>]
    [frame_spec]
)
\end{lstlisting}

\begin{enumerate}
    \item \textbf{Función agregada:}
    \begin{itemize}
        \item Por ejemplo: \texttt{SUM}, \texttt{AVG}, \texttt{ROW\_NUMBER}, \texttt{RANK}, \texttt{LEAD}, \texttt{LAG}, etc.
    \end{itemize}
    \item \textbf{Cláusula \texttt{OVER}:}
    \begin{itemize}
        \item Define el conjunto de filas sobre el que se aplica la función.
    \end{itemize}
    \item \textbf{\texttt{PARTITION BY:}}
    \begin{itemize}
        \item Divide las filas en subconjuntos independientes.
        \item Similar a \texttt{GROUP BY}, pero sin colapsar los datos.
    \end{itemize}
    \item \textbf{\texttt{ORDER BY:}}
    \begin{itemize}
        \item Ordena las filas dentro de cada partición.
        \item Necesario para cálculos como acumulativos o rankings.
    \end{itemize}
    \item \textbf{\texttt{frame\_spec:}}
    \begin{itemize}
        \item Define el rango de filas sobre el que se aplica la función.
        \item Opcional y avanzado.
    \end{itemize}
\end{enumerate}

\section{Ejemplos de window functions}

\subsection{Ranking y análisis relativo}

Determinar el ranking de empleados por salario dentro de cada
departamento.

\begin{lstlisting}[language=SQL]
SELECT employee_id, department_id, salary,
    RANK() OVER (PARTITION BY department_id ORDER BY salary DESC) as rank
FROM employees;
\end{lstlisting}

\textbf{Funciones de ranking:}

\begin{itemize}
    \item \texttt{RANK()} devuelve el puesto de cada fila con posibles <<saltos>> en caso de empate.
    \item \texttt{DENSE\_RANK()} devuelve el puesto sin saltos.
    \item \texttt{ROW\_NUMBER()} devuelve un número de fila único, por lo que no hay empates.
\end{itemize}

\subsection{Cálculos acumulativos}

Calcular el total acumulado de ventas por cliente, ordenado por fecha.

\begin{lstlisting}[language=SQL]
SELECT customer_id, sale_date, amount,
    SUM(amount) OVER (PARTITION BY customer_id ORDER BY sale_date) as cumulative_total
FROM sales;
\end{lstlisting}

\subsection{Comparaciones entre filas}

Calcular la diferencia entre el salario actual y el del empleado anterior
en el departamento.

\begin{lstlisting}[language=SQL]
SELECT employee_id, department_id, salary,
    salary - LAG(salary) OVER (PARTITION BY department_id ORDER BY salary) as salary_diff
FROM employees;
\end{lstlisting}

\textbf{Funciones de comparación:}

\begin{itemize}
    \item \texttt{LEAD()} devuelve el valor de la siguiente fila.
    \item \texttt{LAG()} devuelve el valor de la fila anterior.
\end{itemize}

\subsection{Agregados no destructivos}

Mostrar el salario promedio del departamento junto a cada empleado.

\begin{lstlisting}[language=SQL]
SELECT employee_id, department_id, salary,
    AVG(salary) OVER (PARTITION BY department_id) as avg_salary
FROM employees;
\end{lstlisting}

\subsection{Series temporales}

Calcular el cambio porcentual de ventas mes a mes.

\begin{lstlisting}[language=SQL]
SELECT sale_date, amount,
    (amount - LAG(amount) OVER (ORDER BY sale_date)) / LAG(amount) OVER (ORDER BY sale_date) * 100 AS pct_change
FROM sales;
\end{lstlisting}

\texttt{LAG(amount)} recupera el valor del mes anterior para calcular
la diferencia y el cambio porcentual.

\section{Mejorar el rendimiento con window functions}

\begin{itemize}
    \item \textbf{Usa índices:}
    \begin{itemize}
        \item Las columnas en las cláusulas \texttt{PARTITION BY} y \texttt{ORDER BY} pueden beneficiarse de índices.
    \end{itemize}
    \item \textbf{Evita recalcular datos repetidamente:}
    \begin{itemize}
        \item Las funciones de ventana calculan resultados basados en un
        solo conjunto de datos, mejorando la eficiencia comparado con subconsultas.
    \end{itemize}
\end{itemize}
