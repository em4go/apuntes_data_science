\chapter{Creación de bases de datos}

La creación de bases de datos y tablas es un proceso
crítico que afecta al rendimiento, la organización y la
escalabilidad de las apliaciones. En este capítulo se
explican los conceptos básicos de la creación de bases
de datos y tablas en MySQL.

\section{Crear una base de datos}

La base de datos es el contenedor lógico que almacena tablas,
vistas, procedimientos almacenados, funciones y otros objetos.

\subsection{Sintaxis}

\begin{lstlisting}[language=SQL]
CREATE DATABASE nombre_base_datos;
[CHARACTER SET nombre_juego_caracteres]
[COLLATE nombre_colacion];
\end{lstlisting}

\vspace{10pt}

\subsection{Mejores prácticas}

\begin{itemize}
    \item \textbf{Utilizar nombres descriptivos.} Evitar nombres genéricos como \texttt{db}, \texttt{test}, etc.
    \item \textbf{Utilizar minúsculas.} Los nombres de bases de datos y tablas son case insensitive en Windows, pero case sensitive en Linux por defecto.
    \item \textbf{Utilizar guiones bajos o guiones.} Evitar espacios y otros caracteres especiales.
    \item \textbf{Utilizar un juego de caracteres y colación adecuados.} Se recomienda utilizar \texttt{utf8mb4} y \texttt{utf8mb4\_unicode\_ci} respectivamente.
    \item \textbf{Utilizar \texttt{IF NOT EXISTS}.} Evitar errores si la base de datos ya existe.
\begin{lstlisting}[language=SQL]
CREATE DATABASE IF NOT EXISTS tienda
CHARACTER SET utf8mb4
COLLATE utf8mb4_unicode_ci;
\end{lstlisting}
    \item \textbf{Versionamiento.} Incluye un esquema de versionamiento en la documentación o prefijos para reflejar cambios.
\end{itemize}

\section{Crear una tabla}

Las tablas son la estructura principal donde se almacenan los datos.

\subsection{Sintaxis}

\begin{lstlisting}[language=SQL]
CREATE TABLE nombre_tabla (
    nombre_columna1 tipo_dato1 [restricciones],
    nombre_columna2 tipo_dato2 [restricciones],
    ...
    [restricciones_tabla]
);
\end{lstlisting}

\subsection{Buenas prácticas al diseñar tablas}

\subsubsection{Definir la clave primaria}

Asegúrate de que cada tabla tenga una clave primaria
(\texttt{PRIMARY KEY}) que identifique de forma única
cada registro.

Las claves primarias suelen ser una columna auto incremental
(\texttt{INT AUTO\_INCREMENT}) o un identificador único
(UUID \texttt{CHAR(36)}).

\begin{lstlisting}[language=SQL]
CREATE TABLE employees (
    employee_id INT AUTO_INCREMENT PRIMARY KEY,
    name VARCHAR(100) NOT NULL,
    department_id INT,
    hire_date DATE
);
\end{lstlisting}

\subsubsection{Elige el tipo de datos correcto}

Usa los tipos de datos adecuados para cada columna.
Algunos tipos de datos comunes son:

\begin{itemize}
    \item \texttt{INT} para números enteros.
    \item \texttt{DECIMAL} para números decimales.
    \item \texttt{VARCHAR} para cadenas de texto.
    \item \texttt{DATE} para fechas.
    \item \texttt{ENUM} para listas de valores.
    \item \texttt{TEXT} para texto largo.
    \item \texttt{BLOB} para datos binarios.
\end{itemize}

Algunos tipos de datos específicos pueden ayudar a ahorrar espacio:
\begin{itemize}
    \item \texttt{TINYINT} para números pequeños. (0 a 255, 1B)
    \item \texttt{SMALLINT} para números medianos. (-32,768 a 32,767, 2B)
    \item \texttt{VARCHAR} para cadenas de texto cortas en lugar de \texttt{TEXT}.
    \item \texttt{DECIMAL} para números decimales exactos en lugar de \texttt{FLOAT} o \texttt{DOUBLE}. \cite{decimalVsFloat}
\end{itemize}

\subsubsection{Normaliza la base de datos}

La normalización es un proceso que se utiliza para organizar
una base de datos en tablas y columnas para reducir la redundancia
y mejorar la integridad de los datos. Véase \ref{ch:normalizacion}.

Se resume en dividir los datos en varias tablas para eliminar
redundancia y usar claves ajenas (\texttt{FOREIGN KEY}) para
vincular las tablas.

\subsubsection{Utiliza restricciones}

Las restricciones (\texttt{CONSTRAINTS}) son reglas que se
aplican a las columnas de una tabla. Algunas restricciones
comunes son:

\begin{itemize}
    \item \texttt{NOT NULL} para evitar valores nulos.
    \item \texttt{UNIQUE} para valores únicos.
    \item \texttt{DEFAULT} para valores por defecto.
    \item \texttt{CHECK} para validar valores.
    \item \texttt{FOREIGN KEY} para referencias a otras tablas.
\end{itemize}

\begin{lstlisting}[language=SQL]
CREATE TABLE departments (
    department_id INT PRIMARY KEY,
    name VARCHAR(100) NOT NULL UNIQUE,
    budget DECIMAL(10, 2) DEFAULT 0.00
);
\end{lstlisting}

\subsubsection{Usa índices para mejorar el rendimiento}

Define índices en columnas que se usen frecuentemente en
consultas o en cláusulas \texttt{JOIN}.

\begin{lstlisting}[language=SQL]
CREATE INDEX idx_department_name ON departments(name);
\end{lstlisting}

\subsubsection{Nombres claros y consistentes}

Usa nombres descriptivos y en minúsculas, separados por
guiones bajos (\textit{snake case}). Por ejemplo:
\texttt{employee\_details}, \texttt{order\_items}.

\section{Relaciones entre tablas}

Las relaciones entre tablas se establecen mediante claves
ajenas (\texttt{FOREIGN KEY}). Estas aseguran la integridad
referencial y permiten realizar consultas que unen datos
de varias tablas.

\begin{lstlisting}[language=SQL]
CREATE TABLE orders (
    order_id INT AUTO INCREMENT PRIMARY KEY,
    customer_id INT NOT NULL,
    order_date DATE,
    FOREIGN KEY (customer_id) REFERENCES customers(customer_id)
    ON DELETE CASCADE
    ON UPDATE CASCADE
);
\end{lstlisting}

\subsection{Tipos de relaciones}

\begin{itemize}
    \item \textbf{Uno a uno (1:1).} Cada fila de una tabla se relaciona con una fila de otra tabla.
    \item \textbf{Uno a muchos (1:N).} Cada fila de una tabla se relaciona con varias filas de otra tabla.
    \item \textbf{Muchos a muchos (N:M).} Varias filas de una tabla se relacionan con varias filas de otra tabla.
\end{itemize}

\subsection{Buenas prácticas}

\begin{itemize}
    \item Usa \texttt{ON DELETE CASCADE} y \texttt{ON DELETE SET NULL} según sea necesario.
    \item Define índices automáticamente en columnas con claves ajenas para acelerar consultas \texttt{JOIN}.
\end{itemize}

\section{Indexación}

Los índices son estructuras de datos que mejoran la velocidad
de las consultas al permitir a la base de datos buscar
rápidamente los registros.

\subsection{Tipos de índices}

\begin{itemize}
    \item \textbf{Índice primario.} Se crea automáticamente al definir una clave primaria.

    \item \textbf{Índice único.} Garantiza que no haya valroes duplicados en una columna.
    En MySQL, se crea automáticamente al definir una restricción \texttt{UNIQUE}.
\begin{lstlisting}[language=SQL]
CREATE UNIQUE INDEX idx_email on users(email);
\end{lstlisting}

    \item \textbf{Índice múltiple.} Se crea en varias columnas para acelerar consultas que usan esas columnas.
\begin{lstlisting}[language=SQL]
CREATE INDEX idx_id_date on orders(order_id, order_date);
\end{lstlisting}
\end{itemize}

\subsection{¿Cuándo crear índices?}

\begin{itemize}
    \item En columnas que aparecen frecuentemente en cláusulas \texttt{WHERE} o \texttt{JOIN}.
    \item En columnas usadas para ordenar (\texttt{ORDER BY}) o agrupar (\texttt{GROUP BY}).
\end{itemize}

\textbf{Nota:} No abuses de los índices, ya que pueden ralentizar
las operaciones de escritura y ocupar espacio en disco.

\section{Partición de tablas}

La partición de tablas es una técnica que divide una tabla
en fragmentos más pequeños para mejorar el rendimiento y
la administración de la base de datos.

Véase \ref{ch:particiones_tablas}.

\section{Ejemplo completo}

\begin{examplebox}{Ejemplo de creación de base de datos}
    Crea una base de datos para almacenar empleados y departamentos.
\end{examplebox}

\vspace{10pt}

\begin{lstlisting}[language=SQL]
CREATE DATABASE IF NOT EXISTS company
CHARACTER SET utf8mb4
COLLATE utf8mb4_unicode_ci;

USE company;

CREATE TABLE employees (
    employee_id INT AUTO_INCREMENT PRIMARY KEY,
    name VARCHAR(100) NOT NULL,
    department_id INT,
    hire_date DATE,
    salary DECIMAL(10, 2) DEFAULT 0.00,
    FOREIGN KEY (department_id) REFERENCES departments(department_id)
    ON DELETE SET NULL
    ON UPDATE CASCADE
)

CREATE TABLE departments (
    department_id INT PRIMARY KEY,
    name VARCHAR(100) NOT NULL UNIQUE,
    budget DECIMAL(10, 2) DEFAULT 0.00
);

CREATE INDEX idx_department_name ON departments(name);
\end{lstlisting}