\chapter{Tema 2: Optimización matemática}

\section{Clasificación, regresión y optimización}

Podemos definir un modelo como una función parametrizada por $\Theta$,
que es un conjunto de parámetros.

$$ \mathcal{F} = \{ f_\Theta : X \rightarrow Y, \quad \Theta \in \mathbb{R}^D \} $$

Donde $X$ es el espacio de entrada y $Y$ el espacio de salida.

En el caso de la clasificación, $Y$ es un conjunto finito de etiquetas

$$ Y = \{1, 2, \ldots, C\} $$

En el caso de la regresión, $Y$ es un conjunto continuo.

$$ Y \equiv \mathbb{R} $$

Y típicamente, $X \equiv \mathbb{R}^d$

De esta forma, encontramos dos etapas en el ciclo de vida de un modelo:

\begin{enumerate}
    \item Entrenamiento: Ajustar los parámetros del modelo dado un conjunto de datos de entrenamiento.
    $$\text{Dado } S \subset X \times Y \rightarrow \text{Estimar } \Theta \in \mathbb{R}^D$$
    \item Evaluación: Medir la calidad del modelo en un conjunto de datos de prueba.
    $$\text{Dados } \Theta y x \in X \rightarrow \text{Estimar } y \in Y = f_\Theta (x)$$
\end{enumerate}